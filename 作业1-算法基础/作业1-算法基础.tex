\documentclass{ctexart}

\usepackage{mathtools}
\usepackage{ntheorem}
\usepackage{graphicx}
\usepackage{subfig}  
\usepackage{geometry}
\usepackage{booktabs}
\geometry{a4paper}
\usepackage[ruled,linesnumbered]{algorithm2e}

\title{高级算法设计与分析作业1\\算法基础}
\author{傅彦璋\ 23S003008}
\date{2024.5.6}


\newtheorem{definition}{定义}

\pagestyle{plain}
\begin{document}

\maketitle

\section*{1}

求$a$和$b$($a \ge b$)的最大公约数的欧几里得算法描述如$GCD(a,b)$。 算法的输入规模为$\log_2 a + \log_2 b$。


\begin{algorithm}[H]
  \SetAlgoLined
  \KwData{$a,b$满足$a\ge b$}
  \KwResult{$gcd(a,b)$}
  $r \leftarrow a \mod b$  \;
  \uIf{$r \neq 0$}{
    \Return $GCD(b,r)$
  }
  \Else{
    \Return $b$
  }
  \caption{$GCD(a,b)$}
\end{algorithm}

记算法第$k$次递归调用$GCD(a,b)$时,$b$的值为$b_k$。算法第1行的$r = a\mod b$意味着$r\le \frac{a}{2}$,所以,对任意$k$,一定有$b_{k+2}\le \frac{b_k}{4}$。当某一次递归调用$GCD(a,1)$,即$b=1$的时候,算法一定停止。所以,算法的递归调用次数不超过$\log_2 b$。故求余操作不超过$\log_2 b$次,赋值操作不超过$3\log_2 b$次。

我们知道,计算$a \mod b$操作的时间代价是$O(\log_2 a)$的。所以所有求余操作产生的时间代价是$O(\log_2 a \cdot \log_2 b)\in O(\log^2_2 a)$的。

于是,对于规模为$O(n)$的输入规模,算法的时间复杂度为$O(n^2)$。





\section*{2}
\subsection*{(a)}

算法由两重循环构成。外层循环最多循环$n-1$次(变量$i$从$2$增长到$n$即停止),是有限的次数,一定会停止。在每个外层循环中,内层循环最多循环$i-1$次(变量$j$从$i-1$下降到$1$即停止),而$i-1 < n$,也是有限的。所以算法在有限步内必然停止。


\subsection*{(b)}
用循环不变量方法证明算法的正确性:

循环变量为$i$,初始时,$i=2$,$A[1]$来自输入且有序。当循环体执行前,若$A[1,2,...,i-1]$来自输入且有序,则循环体执行一遍后,$A[1,2,...,i]$来自输入且有序。终止时,$i=n+1$,故循环体执行最后一遍后,$A[1,2,...,n]$来自输入且有序。


\subsection*{(c)}

最好情况:初始序列是有序的,算法的第5、6行不会执行。赋值操作仅出现在第2、3、7行。显然赋值操作次数和比较操作次数都是$O(n)$的。

最坏情况:初始序列是逆序的。对于每个外层循环,算法的内层循环都要执行$i-1$遍。赋值操作次数和比较操作次数都是$O(n^2)$的。

平均情况:所有输入服从均匀分布,我们知道,整个序列中逆序对数平均为$\frac{n(n-1)}{4}$,那么,在算法5、6行发生的赋值操作次数为$\frac{n(n-1)}{2}$,那么相对应的,在算法第4行发生的比较操作次数不低于$\frac{n(n-1)}{2}$。考虑上算法第2、3、7行的操作,总的赋值操作次数是$\frac{n^2}{2} + \frac{5n}{2}$。总之,比较次数和赋值次数皆是$O(n^2)$的。







\section*{3}

算法的输入为整数$n$,输入规模为$n$的规模,在二进制模型下即$\log_2 n$,总之是$O(\log n)$的规模。循环将执行最多$n^{1/2}$次,即$\left( 2^{\log_2 n}\right) ^{1/2}$次。对于输入规模$n$,算法的基础操作次数是$2^{n/2}$,是指数时间算法。


\section*{4}

算法的输入为整数$n$,输入规模为$n$的规模,在二进制模型下即$\log_2 n$,总之是$O(\log n)$的规模。当$n\ge 2$的时候,算法执行基本操作加法的次数是$n-1$次,即$2^{\log_2 n}-1$。即,对输入规模$x$,算法基本操作的次数是$2^x-1$,所以算法是指数时间算法。



























\end{document}  