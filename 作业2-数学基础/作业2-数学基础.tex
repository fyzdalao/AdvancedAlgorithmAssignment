\documentclass{ctexart}

\usepackage{mathtools}
\usepackage{ntheorem}
\usepackage{graphicx}
\usepackage{subfig}  
\usepackage{geometry}
\usepackage{booktabs}
\geometry{a4paper}
\usepackage[ruled,linesnumbered]{algorithm2e}

\title{高级算法设计与分析作业2\\数学基础}
\author{傅彦璋\ 23S003008}
\date{2024.5.3}


\newtheorem{definition}{定义}

\pagestyle{plain}
\begin{document}

\maketitle


\section*{1}
反证,假设不为空集。任取$f(n)\in o(g(n)) \cap \omega(g(n)) $.

对任意常数$c$,都有:

因为$f(n)\in o(g(n))$,所以$\exists n_0 $,使得$n>n_0$时,$f(n)<c\cdot g(n)$成立。
因为$f(n)\in \omega(g(n))$,所以$\exists n_1 $,使得$n>n_1$时,$f(n)>c\cdot g(n)$成立。

取$n_2 = max\{n_0,n_1\}$,当$n>n_2$时,$c\cdot g(n)<f(n)<c\cdot g(n)$,这是不可能的。所以$o(g(n)) \cap \omega(g(n)) $是空集。

\section*{2}

记$h(n)=\max \{f(n),g(n)\}$.根据$h(n)$的定义,我们知道,$\exists n_0$,当$n>n_0$时,有$0\le g(n)\le h(n),\ 0\le f(n)\le h(n)$,又有$h(n)\le (f(n)+g(n))$.

总之$\frac{1}{2}\left( f(n)+g(n) \right) \le h(n) \le f(n)+g(n)$.所以$h(n)$与$f(n)+g(n)$是同阶的,即$\max \{f(n),g(n)\} = \Theta(f(n)+g(n))$.


\section*{3}

先证明$\log (n!) = O(n\log n)$:

$\log(n!) = \log 1 + \log 2 + ... + \log n \le \sum_{i=1}^{n}\log n = n\log n = O(n log n)$.

再证明$\log (n!) = \Omega(n\log n)$:
$$
\log (n!) = \sum_{i=1}^n \log i \ge \sum_{i=n/2}^n \log i \ge \sum_{i=n/2}^n \log \frac{n}{2} = \frac{n}{2}\log \frac{n}{2} = \frac{n}{2}\log n - \frac{n}{2}\log 2
$$
当$n>4$时,$\frac{n}{2}\log n - \frac{n}{2}\log 2 \ge \frac{n\log n}{4}$,所以$\log (n!) = \Omega(n\log n)$。

总之,$\log (n!) = \Theta(n\log n)$.



\section*{4}

$$
\begin{aligned}
T(n)&=T\left(\frac{3}{10}n\right)+5n\\
&=T\left( \left(\frac{3}{10}\right)^2 n  \right) + 5n\left(1+\frac{3}{10}\right)\\
&=T\left( \frac{3^k}{10^k} n \right) +5n \left( 1+\frac{3}{10}+...+\left(  \frac{3}{10} \right)^{k-1} \right)\\
\end{aligned}
$$

令$\frac{3^k}{10^k} n =1$,则$n=\left( \frac{10}{3} \right)^k$.

$$
\begin{aligned}
T(n) &= T(1)+5\cdot \left( \frac{10}{3} \right)^k \cdot \left(  \frac{1-\left(\frac{3}{10}\right)^k}{1-\frac{3}{10}} \right)  \\
&= T(1)+ \frac{50}{7} \left( \left( \frac{10}{3}\right) ^k -1 \right)\\
&=T(1) + \frac{50(n-1)}{7}\\
&=O(n)
\end{aligned}
$$


\section*{5}

$$
T(n)=T(\lceil n/2 \rceil )+1 = T(\lceil \frac{n}{2^k} \rceil )+k
$$

令$\frac{n}{2^k}=1$,则$k=\log _2 n$,则

$$
T(n)=T(1)+ \log_2 n = O(\log n)
$$


\section*{6}

$T(n)=3T(n^{1/2})+ \log n$.令$n=2^m$,则$T(2^m) = 3T(2^{m/2})+m$.
令$S(m)=T(2^m)$,则$S(m)=3S(\frac{m}{2})+m$.

由Master定理,$S(m) = \Theta(m^{\log_2 3})$.

$T(n) = S(\log_2 n) = \Theta\left( (\log_2 n)^{\log _2 3}  \right)$.


\section*{7}

$T(n) = 3T(n-1)+2^n$,令$m=2^n$,则$T(\log_2 m) = 3T(\log_2 \frac{m}{2}) + m$.

令$S(m) = T(\log_2 m)$,则$S(m) = 3S(m/2) + m$.

由Master定理,$S(m) = \Theta\left( m^{\log_2 3} \right)$.故$T(n) = S(2^n) = \Theta \left( 2^{n\log_2 3} \right)$.


\section*{8}
$T(n) \le 2T\left(\lfloor 3n/4 \rfloor \right) + n$,由Master定理,$T(n) \in O\left(n^{\log_{4/3} 2} \right)$.

$T(n) \ge 2T\left( \lfloor n/2 \rfloor \right)+n$,由Master定理,$T(n)\in \Omega(n\log n)$.


\section*{9}
$T(n) = 2T(n/4) + n^{1/2}$.由Master定理,$T(n)=\Theta(\sqrt{n}\log n)$.


\section*{10}
$T(n)=T(\lfloor n/2\rfloor ) + n^3$.由Master定理,$3>\log_2 1$,$T(n)=\Theta(n^3)$.


\section*{11}
$T(n)=5T(n/3) + n$.由Master定理,$\log_3 5 > 1$,$T(n) = \Theta(n^{log_3 5})$.


\section*{12}
$T(n)=4T(n/2)+n^2\log n$.根据Master,记$b=2,a=4,f(n)=n^2\log n$,则$f(n)=\Theta(n^{log_b a} \log n )$,所以$T(n)=\Theta(n^{log_b a} \log^2 n)=\Theta(n^{2} \log^2 n)$.


\section*{13}
$T(n)=\sqrt{n}T(\sqrt{n})+n$,则$T(n)/n = T(\sqrt{n})/\sqrt{n} + 1$.

令$S(n)=\frac{T(n)}{n}$,则$S(n)=S(\sqrt{n})+1$.

令$n=2^m$,则$S(2^m)=S(2^{m/2})+1$.令$R(m)=S(2^m)$,则$R(m)=R(m/2)+1$.

由Master定理,$R(m)=\Theta(\log m)$.则$S(n)=\Theta(\log \log n)$.则$T(n)=\Theta(n\log \log n)$.
































\end{document}  