\documentclass{ctexart}

\usepackage{mathtools}
\usepackage{ntheorem}
\usepackage{graphicx}
\usepackage{subfig}  
\usepackage{geometry}
\usepackage{booktabs}
\geometry{a4paper}
\usepackage[ruled,linesnumbered]{algorithm2e}

\title{高级算法设计与分析作业8\\随机算法}
\author{傅彦璋\ 23S003008}
\date{2024.5.1}


\newtheorem{definition}{定义}

\pagestyle{plain}
\begin{document}

\maketitle

\section*{1}

\begin{algorithm}[H]
  \SetAlgoLined
  \KwData{有序序列$a_1,...,a_n$;待查找元素$val$}
  \KwResult{$val$在序列中的位置,如$val$不存在于序列,则位置为$-1$ }

  初始化$l \leftarrow 1,\ r\leftarrow n $\;
  \While{$l \le r$}{
    均匀随机选取整数$k\in [l,r] $ \;
    \If{$a_k=val$}{
      \Return k
    }  
    \If{$a_k > val$}{
      $r\leftarrow k-1$\;
    }
    \If{$a_k < val$}{
      $l\leftarrow k+1$\;
    }
  }  
  \Return \  -1
  \caption{随机折半查找}
\end{algorithm}

随机折半查找算法的期望时间复杂度为$O(logn)$。

证明:

算法经历若干轮迭代,每一轮迭代将问题规模减小,使得问题规模从$n$逐步收缩至$1$。设随机变量$T_n$表示问题规模从$n$缩减到$1$经历的迭代次数。设随机变量$X_n$表示问题规模为$n$的时候,下一轮迭代使问题规模减小的程度。

存在某个单调非减函数$g(n)=E[X_n]$。我们用归纳法证明:$E[T_n]\le \int_1^n \frac{dx}{g(x)} $.记$f(n)=\int_1^n \frac{dx}{g(x)}$。假设对$m<n$,$E[T_m]\le f(m)$成立,往证$E[T_n]\le f(n)$。

$$
\begin{aligned}
E[T_n] &\le 1+E[T_{n-X_n}]\\
&\le 1+ E\left[f(n-X_n)\right]\\
&=1+E\left[\int_{1}^{n-X_n}\frac{dy}{g(y)}\right]\\
&=1+E\left[\int_1^n \frac{dy}{g(y)} -\int_{n-X_n}^n\frac{dy}{g(y)}\right]\\
&\le 1+f(n)-E\left[\int_{n-X_n}^n\frac{dy}{g(y)}\right]\\
&\le 1+f(n)-E\left[\int_{n-X_n}^n\frac{dy}{g(n)}\right]\\
&\le 1+f(n)-\frac{E[X_n]}{g(n)}\\
&=f(n)
\end{aligned}
$$

所以,算法总共迭代轮数的期望$E[T_n]\le \int_1^n \frac{dx}{g(x)}$。现在,我们只需要找到$g(x)$。

接下来证明$g(x)\ge x/4$。考虑一轮迭代,当前问题规模为$m$(即算法第3行$[l,r]$区间的大小为m),下一轮问题规模为$m_{next}$。本次选择的$k$以$1/m$的概率等于$l,l+1,...,r-1$中的任何一个值,并将$[l,r]$划分为两个区间$[l,k-1]$和$[k+1,r]$。运气较坏时,下一轮的问题规模等于这两个区间中较长一个的长度。在这种“运气坏”的情况下,有:
$$
\begin{aligned}
E[m_{next}]&\le \frac{1}{m}\cdot 2 \cdot \sum_{i=\lfloor m/2 \rfloor}^{m-1}i\\
&\le \frac{1}{m} \left( \lfloor m/2 \rfloor + m  \right)\cdot \lceil m/2 \rceil\\
&\le \frac{3m}{4}
\end{aligned}
$$
所以事实上,$E[m_{next}] \le \frac{3m}{4}$.所以$g(x) \ge x/4$。

于是,算法总共迭代轮数的期望$E[T_n]\le \int_1^n \frac{dx}{g(x)} \le 4\int_1^n \frac{dx}{x} = 4\ln n \in O(log n)$。每轮迭代的操作只是随机选择一个位置$k$和比较大小,代价认为是$O(1)$的。所以随机折半查找算法的期望时间复杂度为$O(logn)$。

\section*{2}
\begin{algorithm}[H]
  \SetAlgoLined
  \KwData{z}
  \KwResult{$F(z)$}
  均匀随机抽取$x\in\{0,1,..,n-1\}$\;
  $y\leftarrow (z-x) \mod n$\;
  \Return $(F(x)+F(y)) \mod m$
  \caption{计算$F(z)$}
\end{algorithm}

Algorithm2能够以大于$\frac{1}{2}$的概率返回正确的$F(z)$。

这是因为,对$\forall x,P(F(x)\ is \ correct)=0.8$。于是算法给出正确解的概率
$$
P_{correct} = P(F(x)\ is\ correct\ and\  F(y)\ is \ correct)\ge 0.8^2 = 0.64 > \frac{1}{2}
$$

如果算法运行3次:

如果算法3次给出的解相同,则返回该解。正确的概率$P_1 \ge 1-(1-P_{correct})^3 = 0.953344$.

如果算法2次给出的解相同,另一次不同,则返回2次相同的解。正确的概率$P_2 \ge 1-(1-P_{correct})^2 = 0.8704$.

如果运行3次结果均不同,则返回最后一次的解。正确的概率不低于$0.64$。



\section*{3}
\subsection*{(1)}

反证:假设$I$不是独立集,那么必然存在$u,v \in I$使得$uv \in E$。假设$u$的标签比$v$的小,那么随着算法的运行$u$一定比$v$更早加入集合$I$。当$u$加入集合$I$时,与其相邻的点,包括$v$在内,都会从$S$中删除。于是$v$永远不会被算法第4步取到,也永远不会加入$I$。$v\notin I$与假设矛盾。所以$I$一定是独立集。

\subsection*{(2)}
结论有误。反例:考虑一个有三个点、两条边的连通图。题目中结论显然有误。


\section*{4}
\subsection*{(1)}
拉斯维加斯算法。

\subsection*{(2)}
存在常数$b=0.75$满足题目要求。

证明(事实上证明过程与第1题证明$g(x)\ge \frac{x}{4}$的过程完全一致,只是符号表示不同):

考虑某一轮递归,设此轮问题规模为$m$,随机变量$m_{next}$表示下一轮的问题规模。记本次选择的元素$s$是当前这轮$m$个数中第$x$小的,也就是说$x=|S_1|+1$。

如果$x=k$则下一轮问题规模为0;如果$x<k$则下一轮问题规模为$m-x$,如果$x>k$则下一轮问题规模为$x-1$。所以:
$$
\begin{aligned}
E[m_{next}] &= \frac{1}{m}\left(  0 + \sum_{x=1}^{k-1}(m-x) + \sum_{x=k+1}^m (x-1)  \right)\\
&\le \frac{1}{m} \left( \sum_{x=1}^{m/2}(m-x) + \sum_{x=m/2}^m (x-1) \right)\\
&=\frac{1}{m} \left( \left( m-1+\frac{m}{2} \right)\frac{m}{2} \right)\\
&=\frac{3m}{4}-\frac{1}{2}\\
&< \frac{3m}{4}
\end{aligned}
$$

所以,存在$b=\frac{3}{4}$使得算法递归过程中考虑的集合大小的数学期望不超过$bn$。

\subsection*{(3)}

算法经历若干轮递归迭代,每一轮迭代将问题规模减小,使得问题规模从$n$逐步收缩至$1$。设随机变量$T_n$表示问题规模从$n$缩减到$1$经历的迭代次数。设随机变量$X_n$表示问题规模为$n$的时候,下一轮迭代使问题规模减小的程度。存在某个单调非减函数$g(n)=E[X_n]$。

我们在本次作业的第1题中已经证明,$E[T_n]\le \int_1^n \frac{dx}{g(x)} $。(2)的结论告诉我们$g(n)\ge \frac{n}{4}$。于是$E[T_n]\le 4 \ln n$.第$i$轮迭代考虑的集合大小期望不超过$n\cdot \left(\frac{3}{4}\right)^i$.所以,算法运行时间期望
$$
E[Time]\le \sum_{i=0}^{4\ln n} n\cdot \left(\frac{3}{4}\right)^i = 4n\left( 1-n^{-4 \ln \frac{4}{3}} \right)\in O(n)
$$













































\end{document}  